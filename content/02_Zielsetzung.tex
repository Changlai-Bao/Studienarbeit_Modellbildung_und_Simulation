\chapter{Zielsetzung der Simulation}\label{ch:Zielsetzung}
Die Simulation des Doppelpendels und des Dreifachpendels soll die Dynamik dieser Systeme untersuchen. Ziel ist es, die Bewegung der Pendel zu analysieren und die Auswirkungen der verschiedeneren Methoden der Modellierung und Simulation zu vergleichen. Dazu wird die Darstellung der Ergebnisse in Form von Diagrammen und Animationen dargestellt, um die Bewegung der Pendel besser zu visualisieren.

Diese Arbeit bezieht sich auf mehrere relevante Aspekte der Modellierung und Simulation. Zuerst wird die Modellierung mit verschiedener Taktiken durchgeführt. Die erste Taktik ist die Anwendung des 5-Punkte Schemas, das eine numerische Methode zur Lösung von Differentialgleichungen darstellt. Die zweite Taktik ist die Anwendung des Zustandsraummodells, dazu umformuliert das System in lineares Zustandsraum.

Danach wird die Systemmodellierung in Matlab (ohne Simulink) durchgeführt. Dazu wird das Skript erstellt, das die Differentialgleichungen des Systems aufstellt und löst. Die Ergebnisse der Simulation werden mit den Ergebnissen der Simulink-Simulation verglichen. 

Schließlich wird das Modell auf ein Dreifachpendel erweitert, um die Auswirkungen der zusätzlichen Freiheitsgrade auf das Systemverhalten zu untersuchen. Hier werden die Ergebnisse der Simulation des Dreifachpendels ebenfalls analysiert und mit den Ergebnissen des Doppelpendels verglichen.
 




